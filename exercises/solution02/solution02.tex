\documentclass[
  ngerman,
  DIV=12
]{scrartcl}
\usepackage{babel}
\usepackage{csquotes}

% typography
\usepackage{fontspec}
\usepackage{newpxmath}
\setsansfont{Open Sans}[
  BoldFont={Open Sans Bold},
  ItalicFont={Open Sans Italic}]
\setmonofont[Scale=0.87]{Menlo}
\setmainfont{Palatino}
\linespread{1.15}
\usepackage[factor=1000]{microtype}

% graphics, drawings, etc.
\usepackage{xcolor}
\usepackage{graphicx}
\usepackage[most]{tcolorbox}
\usepackage{tikz}
\usetikzlibrary{shapes.geometric}
\usetikzlibrary{shapes.arrows}
\usetikzlibrary{positioning}
\newtcolorbox{anmerkung}{%
  grow to left by=10pt,
  colback=black!10,
  colframe=white,
  coltitle=black,
  borderline west={4pt}{0pt}{black!30},
  boxrule=0pt,
  boxsep=0pt,
  %breakable,
  enhanced jigsaw,
  title={Anmerkung\par},
  fonttitle={\bfseries},
  attach title to upper={}}

% highlighting, lists, code
\usepackage{soul}
\usepackage{enumitem}
\usepackage{listings}
\lstset{
  basicstyle=\ttfamily,
  numberstyle=\sffamily\footnotesize\color{gray},
  %escapeinside=||,
  keywordstyle=\color{blue!50!black},
  stringstyle=\color{green!50!black}}
\capsdef{////}{\scshape}{.16em}{.4em}{.2em}

% math
\usepackage{amsmath,amssymb}

% nice tables
\usepackage{booktabs}
\newcommand{\tablespacing}[1]{\renewcommand{\arraystretch}{#1}}

% links
\usepackage[
  colorlinks,
  linkcolor={red!50!black},
  citecolor={blue!50!black},
  urlcolor={blue!80!black}
]{hyperref}

\title{Lösungsblatt Nr. 2}
\date{Wintersemester 2018-2019}
\author{Hendrik Beckmann, Ibrahim Osman,\\Patrick Elsen, Tiancheng Chen}
\subject{Computersystemsicherheit}
\publishers{Technische Universität Darmstadt}

\begin{document}
\maketitle

\section*{Hausübung 1: Diffie-Hellman Schlüsselaustausch}

\emph{Alice möchte mit Bob auf sichere Weise kommunizieren. In einer Einführungsveranstaltung zur Kryptographie erfährt Alice von dem Diffie-Hellman Schlüsselaustauschverfahren und möchte dieses dafür verwenden.}

\emph{Finden Sie den vereinbarten Schlüssel.}

Hier kann man ein kleines Ruby-Programm anwenden, um per Brute-Force an den geheimen Parameter $x$ zu kommen.
\begin{lstlisting}[language=ruby,numbers=left]
def compute_pub(g, p, sec)
  (g * sec) % p
end

def compute_key(g, p, pub, sec)
  (pub * sec) % p
end

p = 13
g = 6

pub_x = 2
pub_y = 3

sec_x = (1..p).select{|n| compute_pub(g, p,  n) == pub_x}
p sec_x
exit

k = compute_key(g, p, pub_y, sec_x)

puts "x is #{sec_x} and the key is #{k}."  
\end{lstlisting}
Dieses Programm berechnet als geheimen Parameter $x = 5$ und den Key $k = 9$.


\emph{Alice entscheidet sich um ihren Rechenaufwand zu reduzieren eine additive Gruppe für den Diffie-Hellman-Schlüsselaustausch zu verwenden, d.h. statt $X = g^a$ verwendet sie jetzt $X = g \cdot a$. Begründen Sie, warum das keine gute Idee ist.}

Der Diffie-Hellman Algorithmus besteht  auf der (in annehmbarer Zeit) Unlösbarkeit des diskreten Logarithmus. Der diskrete Logarithmus ist die Umkehrung der diskreten Exponentialfunktion $b^x \mod n$. Also genau die FUnktion, die für die Schlüsselberechnung genutzt wird. Alice's Funktion ist im Vergleich sehr einfach lösbar und somit nicht sicher.

\section*{Hausübung 2: RSA}

Hier kann man wieder ein kleines Ruby-Programm schreiben, um das Ergebnis zu berechnen.
\begin{lstlisting}[language=ruby,numbers=left]
n = 111791377
e = 3

transmitted = [106894622, 54756549, 49966544]

def encrypt(m, e, n)
  (m ** e) % n
end

grades = [10, 13, 17, 20, 23, 27, 30, 33, 37, 40, 50]
alice = 174458

candidates = grades
  .map{|g| [g, "#{alice}#{g}"]}
  .map{|g, m| [g, encrypt(m.to_i, e, n)]}
match = candidates.select{|g, c| transmitted.include? c}.first

puts "Alice's ciphertext war #{match[1]} und sie hat eine #{match[0]}."  
\end{lstlisting}
Dieses Programm gibt aus, dass Alice eine 1,7 hat, weil ihr Ciphertext 54756549 war.

\section*{Hausübung 3: RSA}

\emph{Alice und Bob haben von dem RSA-Verschlüsselungsver- fahren gehört und möchten sich das jetzt gerne etwas genauer ansehen.}

\emph{Gegeben ist ein RSA-Modulus $N = p \cdot q$ mit $N = 77$ und $p = 7, q = 11$ bekannt. Weiterhin ist der public key $pk = (N,e) = (77,6)$ und Alice möchte mit diesem Schlüssel $m1 = 5$ und $m2 = 6$ verschlüsseln. Da Alice aber nicht mehr weiß wie das RSA-Verfahren funktioniert benötigt sie Ihre Hilfe. Berechnen Sie mit Hilfe des RSA-Verfahrens die Verschluüsselungen von m1 und m2.}

Hier kann man ein kleines Ruby-Programm schreiben, um die Nachricht mit RSA zu verschlüsseln.
\begin{lstlisting}[language=ruby,numbers=left]
def encrypt(m, e, n)
  (m ** e) % n
end

messages = [5, 6]
n = 77
e = 6

p messages.map{|m| encrypt(m, e, n)}  
\end{lstlisting}
Dieses Programm gibt als Ergebnis $[71, 71]$ aus.

\emph{Vergleichen Sie die beiden Ciphertexte. Was fällt Ihnen aus? Warum entsteht dieses
Ergebnis?}

Die beiden Ciphertexte sind identisch. Das liegt wahrscheinlich an dem Exponent $e$, der mit $p$ und $q$ nicht kompatibel ist. Besser ist es, wenn man eine Primzahl als Exponent benutzt.

\emph{Bob besitzt den öffentlichen RSA-Schlüssel $(N, e) = (3127, 6)$ mit $N = pq$.
Warum ist dies kein gültiger öffentlicher RSA-Schlüssel? Ändern Sie den ungültigen Anteil von Bobs öffentlichem Schlüssel minimal so ab, dass er gültig wird, und berechnen Sie anschließend den zugehörigen privaten Schlüssel d für ihn.}

Für einen gültigen Schlüssel wählt man zwei Primzahlen $p$ und $q$ und berechnet $N = pq$. Dann wählt man ein $e$ für das gilt :$ggT(e, \lambda(N)) = 1$. Dies ist bei Bobs' Schlüssel nicht gegeben, da der $ggT(6, 2016) = 2$ ist. 

Man kann die 6 zu einer 5 machen, dessen $ggT(5, 3016)$ ist 1. Damit ist der Schlüssel gültig. Den privaten Schlüssel berechnet man mit dem erweiterten Euklidischen Algorithmus.

\end{document}